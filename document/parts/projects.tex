%!TEX root = ../../cv-angelo-filaseta.tex
% chktex-file 8
\cvsection{ALCUNI PROGETTI IN CUI HO CONTRIBUITO}

\cvitem{\cvdurationstyle{Novembre 2020}}{
	\cvtitle{\href{https://alchemistsimulator.github.io/}{Alchemist}}\\
	Un simulatore stocastico che permette la simulazione di scenari inerenti la computazione pervasiva,
	aggregata ed ispirata alla natura.
	\begin{itemize}[leftmargin=*]
		\item \textbf{Strumenti e Tecnologie}: Gradle · Kotlin Multiplatform · React in Kotlin/JS · GitHub Actions
		\item \textbf{Contributo}: Sviluppo di un componente Renderer flessibile ed in grado di essere spostato
		dinamicamente e a run-time all'interno di un sistema distribuito che utilizza tecnologie eterogenee.
	\end{itemize}
}

\cvitem{\cvdurationstyle{Novembre 2021 - Settembre 2022}}{
	\cvtitle{\href{https://tale152.github.io/brittany/}{Brittany}}\\
	Un sistema volto a semplificare ed automatizzare la gestione di un sistema di serre grazie a sensori ed attuatori, attraverso l’esecuzione, all’interno di una rete privata, di un coordinatore ad agenti.
    \begin{itemize}[leftmargin=*]
		\item \textbf{Strumenti e Tecnologie}: Java · C++ · JaCaMo Framework · Node.js · Express · React.js · Redux ·
		Mongoose · Gradle · PlatformIO · Docker · JSON-LD  
		\item \textbf{Contributo}: Sviluppo della struttura dei componenti fisici al fine di garantire un alto livello
		di flessibilità. Le funzionalità specifiche vengono fornite in base ai tipi di sensori e attuatori disponibili
		e al luogo in cui si trova il dispotivio, utilizzando la filosofia ed alcuni standard del W3C Web Of Things.
	\end{itemize}
}
