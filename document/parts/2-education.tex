\cvsection{FORMAZIONE}

\cvitem{
	\cvdurationstyle{Settembre 2020 -- Marzo 2023}
}{
	\cvtitle{Laurea Magistrale in Ingegneria e Scienze Informatiche L-31}
	
	Università di Bologna (UniBo), Campus di Cesena (FC) - Emilia Romagna\newline
	Tesi Magistrale:
	\begin{itemize}[leftmargin=*]
		\item \textbf{Titolo}: Distributed monitoring and control with dynamic offloading: the case of the Alchemist 
		Simulator
		\item \textbf{Tecnologie utilizzate}: Gradle, Kotlin Multiplatform.
		\item \textbf{Valutazione}: 110L
	\end{itemize}
}

\cvitem{
	\cvdurationstyle{Settembre 2017 -- Ottobre 2020}
}{
	\cvtitle{Laurea Triennale in Ingegneria e Scienze Informatiche L-31}

	Università di Bologna (UniBo), Campus di Cesena (FC) - Emilia Romagna\newline
	Tesi Triennale:
	\begin{itemize}[leftmargin=*]
		\item \textbf{Titolo}: Sviluppo di un'Infografica Responsive per promuovere processi di 
		dematerializzazione.
		\item \textbf{Tecnologie utilizzate}: Stack MERN, Twitter Bootstrap, SVG standard.
  		\item \textbf{Valutazione}: 97
	\end{itemize}
}

\cvitem{
	\cvdurationstyle{Settembre 2013 -- Giugno 2017}
}{
	\cvtitle{Diploma di maturità tecnica}\\
	Istituto Tecnico Industriale (ITI) Don Luigi Orione, Fano (PU) - Marche\newline
	Tesi di maturità:
	\begin{itemize}[leftmargin=*]
		\item \textbf{Titolo}: New Generation Arcade Cabinet
		\item \textbf{Tecnologie utilizzate}: UNIX
  		\item \textbf{Valutazione}: 100/100
	\end{itemize}

}

\cvitem{
	\cvdurationstyle{Maggio 2017}
}{
	\cvtitle{Attestato di qualifica professionale}\\
	Corso 195136 - Analista Informatico\newline
	Corso approvato con decreto P.F Presidio Formazione e Servizi per l'impiego (PU)(AN).
	\begin{itemize}[leftmargin=*]
		\item \textbf{Contenuti}: Salute e Sicurezza, Competenze relazionali, Problem Solving, Analisi Logica, Reti 
		Informatiche, Tecnologie Informatiche, Informatica.
  		\item \textbf{Durata}: 600 ore
  		\item \textbf{Valutazione}: 95/100
	\end{itemize}
}
