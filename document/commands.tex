%!TEX root = ../cv-angelo-filaseta.tex
\newcommand\resume[2]{%
  \ifnum#1>#2
    $#1 > #2$
  \else
    \ifnum#1<0
      $#1 < 0$
    \else
      \ifnum#2<0
        $#2 < 0$
      \else
        \tikz{%
        \ifx#20
        \else
          \foreach\i in {1,\ldots,#2} {
            \filldraw[black!20] (\i ex,0) circle (0.4ex);
          };
        \fi
        \ifx#10
        \else
          \foreach\i in {1,\ldots,#1} {
            \filldraw[black] (\i ex,0) circle (0.4ex);
          };
        \fi
        }
      \fi
    \fi
  \fi
}


\newcommand{\cvpersonalinfo}[2]{
    \begin{minipage}[t]{\cvleftcolumnwidth}
        \vspace{0mm}
        \raggedleft#1
    \end{minipage}
    \hspace{\cvcolumngapwidth}
    \begin{minipage}[t]{\cvrightcolumnwidth}
        \vspace{0mm}
        #2
    \end{minipage}
    \vspace{\cvafteritemskipamount}
}

\newcommand{\cvname}[1]{
    \cvnamestyle{#1}
    \vspace{\cvafternameskipamount}
}

\newcommand{\cvpersonalinfolinewithicon}[3]{
    \raisebox{.5\fontcharht\font`E-.5\height}{\includegraphics[#1]{#2}}
    #3
    \vspace{\cvafterpersonalinfolineskipamount}
}

\newcommand{\cvsection}[1]{
    \\\begin{minipage}[t]{\cvleftcolumnwidth}
        \raggedleft\cvsectionstyle{#1}
    \end{minipage}
    \hspace{\cvcolumngapwidth}
    \begin{minipage}[t]{\cvrightcolumnwidth}
        \textcolor{cvrulecolor}{\rule{\cvrightcolumnwidth}{0.3mm}}
    \end{minipage}
    \vspace{\cvaftersectionskipamount}
}

\newcommand{\cvitem}[2]{
    \begin{minipage}[t]{\cvleftcolumnwidth}
        \raggedleft#1
    \end{minipage}
    \hspace{\cvcolumngapwidth}
    \begin{minipage}[t]{\cvrightcolumnwidth}
        \setlength{\parskip}{\cvparskip} #2
    \end{minipage}
    \vspace{\cvafteritemskipamount}
}

\newcommand{\cvtitle}[1]{
    \cvtitlestyle{#1}
    \vspace{\cvaftertitleskipamount}
	\vspace{-\cvparskip}
}

\newcommand*{\TakeFourierOrnament}[1]{{%
\fontencoding{U}\fontfamily{futs}\selectfont\char#1}}
\newcommand*{\danger}{\TakeFourierOrnament{49}}

\newcommand{\boxed}[1]{
  \tikz[baseline= (X.base)]
  \node[draw=black,fill=gray!10,semithick,rectangle,inner sep=3pt, rounded corners=2pt] (X) {#1};
}

% RE-definition of bibliography style to output in a suitable format for CV
\defbibenvironment{bibliography}
  {\list
     {}
     {%
      \setlength{\labelwidth}{0pt}%
      \setlength{\leftmargin}{0pt}%
      \setlength{\labelsep}{0pt}%
      \addtolength{\leftmargin}{0pt}%
      \setlength{\itemsep}{\bibitemsep}%
      \setlength{\parsep}{\bibparsep}}%
      \renewcommand*{\makelabel}[1]{##1\hss}}
  {\endgilist}
  {\item}
